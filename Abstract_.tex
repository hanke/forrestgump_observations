% Abstracts should be up to 300 words and provide a succinct summary of the
% article. Although the abstract should explain why the article might be
% interesting, care should be taken not to inappropriately over-emphasise the
% importance of the work described in the article. Citations should not be used
% in the abstract, and the use of abbreviations should be minimized.
Here we present a dataset with a description of portrayed emotions in the movie
"Forrest Gump". A total of 12 observers independently annotated emotional
episodes regarding their temporal location and duration. The nature of an
emotion was characterized with basic attributes, such as arousal and valence,
as well as explicit emotion category labels. In addition, annotations include a
record of the perceptual evidence indicating the presence of an emotion. Two
variants of the movie were annotated separately: 1) an audio-movie version of
Forrest Gump that has been used as a stimulus for the acquisition of a large
\href{http://studyforrest.org}{public functional brain imaging dataset}, and
2)~the original audio-visual movie. We present reliability and consistency
estimates that suggest that both stimuli can be used to study visual and
auditory emotion cue processing in real-life like situations. Raw annotations
from all observers are publicly released in full in order maximize their
utility for a wide range of applications, and possible future extensions. In
addition, aggregate time series of inter-observer agreement with respect to
particular attributes of portrayed emotions are provided to facilitate adoption
of these data.
