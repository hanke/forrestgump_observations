\section*{Background} 
%The format of the main body of the article is flexible: it should be concise and in the format most appropriate to displaying the content of the article.

%A brief summary of how this work was motivated and how it links to existing and future work.

Recently, we have published a dataset with high-resolution functional magnetic
resonance (fMRI) data of 20 participants listening to a two-hour audio-movie of
the Hollywood feature film ``Forrest Gump'' \cite{HBI+14}. Using prolonged
complex naturalistic stimuli, such as this one, is one approach to study
cognitive processing in situations that resemble real-life contexts more
closely than controlled laboratory settings typically employed to investigate
individual cognitive functions in isolation \cite{HH2012}. However,
multidimensional stimuli and the resulting lack of experimental control can
make it harder to isolate the intervening parameters \cite{HH2012}.  The goal
of this study was to extend the information about this movie stimulus in order
to enable further analysis of the published brain imaging data, as well as to
investigate the utility of this particular stimulus for future studies and
additional data acquisitions. To this end we focused on a highly relevant
aspect of social cognition that is, at the same time, difficult to infer from
an audio-visual stimulus by means of computer algorithms: portrayed emotions.

The presentation of movies (clips) to elicit emotional responses is an
established component of emotion research, in particular with respect to more
differentiated emotional states, such as remorse or pride, that go beyond
primary emotions, like fear, in complexity and time-scale \cite{GL1995}. Emotion
cues in movies can be manifold: facial expressions (e.g.
smiling) \cite{Ekm1992b}, verbal cues (e.g. swearing), voice characteristics
(e.g. trembling voice) \cite{EVS+2009}, or even context cues that can be used to
reason about emotional aspects of a situation based on abstract causal
principles rather than direct perceptual cues \cite{SS2014}.  Moreover, the
emotional response to a stimulus in a real-life setting is further dependent on
additional factors. For example, observing a facial expression of sadness may
yield an emotional response of pity or satisfaction, depending on whether the
person is a friend or a punished offender of social norms. A smiling face is
not an expression of happiness when the larger context identifies it as a
strategy to avoid unpleasant social interactions (e.g. a ``fake'' smile).
Consequently, labeling portrayed emotions in feature films is a task that
requires human observers to perform complex judgements. Moreover, emotions portrayed in
movies are typically performed by actors that mimic a target expression, to the
best of their abilities, which does not necessarily reflect their true emotional
status.  Consequently, individual indicators of emotions may be missing,
exaggerated, or even contradictory in comparison to ``real'' emotions.

Two groups of frameworks for systematic description of emotions are
distinguished in the literature: a) schemes with discrete emotions (labels) and
b) dimensional models \cite{GW2007}.  Models using discrete emotion labels vary
considerably in the number of differentiated emotion states. Many theories
assume few basic, innate emotions \cite{Ekm1992a}, others discriminate up to
36 affective categories \cite{Sch2005}.  Dimensional models, for example the
circumplex model, locate different emotional states in a two-dimensional
space, commonly using the axes arousal and valence \cite{Rus1980}.

Current emotion assessment tools, like the Facial Action Coding System (FACS),
or the Specific Affect Coding System (SPAFF) \cite{CG2007}, attempt to combine
both approaches. However, they typically involve following complex instructions
regarding the interpretation of facial expressions or other physical and
cultural indicators of emotion \cite[p. 281]{CG2007} and, thus, require an
intensive training of observers. While these tools provide reliable procedures
for rating emotions, they are not very intuitive and consequently only
accessible to a selected group of experts. 

In this study, the primary goal was not to generate an objective labeling of
portrayed emotions in the Forrest Gump movie.  Congruent with our goal to
enable further studies of already published brain imaging data, we rather aimed
at producing a description of the emotional content of the Forrest Gump movie
stimulus as it was likely perceived by the participants in our past and future
brain imaging studies (potentially biased by age, native language, or
education). Therefore, our approach was to collect data from multiple observers
that stem from the same student population, using a simplified procedure
that does not require extensive training.

Portrayed emotions where independently annotated for the audio-movie version of
``Forrest Gump'' (used in~\cite{HBI+14}) and for the original audio-visual
movie, in order to obtain information on how emotional content in both stimulus
variants differs.

The resulting dataset of annotations of portrayed emotions combines a
dimensional rating with a categorical labeling of emotions and a description of
their associated perceptual evidence. In the following, we provide evidence
that our procedure yielded a reliable description that can be used to segment
the movie into episodes of portrayed emotions. Moreover, we show that the
time-course of inter-observer agreement with respect to many aspects of
portrayed emotion is a reliable measure of their perceptual ambiguity
throughout the movie.

\subsection*{Possible applications}

In combination with the already publicly available brain imaging data, these
annotations form a two-hour high-resolution fMRI measurement for auditory
emotion cue processing from 20 participants. With the addition of a future
publication of brain imaging data recorded from a stimulation with the
audio-visual movie, the full dataset will enable comparative studies
investigating the processing of rich emotional stimuli via different sensory
pathways. Moreover, these new annotations of portrayed emotions are another
step towards a comprehensive description of this reproducible movie
stimulus\cite{HBI+14} and improve its general utility for independent studies
with a focus on the perception of emotion in naturalistic settings.
