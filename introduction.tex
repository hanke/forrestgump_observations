\section*{Background} 
%The format of the main body of the article is flexible: it should be concise and in the format most appropriate to displaying the content of the article.

%A brief summary of how this work was motivated and how it links to existing and future work.

Recently, we have published a dataset with high-resolution functional magnetic
resonance (fMRI) data of 20 participants listening to a two-hour audio-movie of
the Hollywood feature film ``Forrest Gump'' \cite{HBI+14}. Using prolonged
complex naturalistic stimuli, such as this one, is one approach to study
cognitive processing in situations that resemble real-life contexts more
closely than controlled laboratory settings typically employed to investigate
individual cognitive functions in isolation \cite{HH2012}. However,
multidimensional stimuli and the resulting lack of experimental control can
make it harder to isolate the intervening parameters \cite{HH2012}. Here we
want to lay the groundwork for an attempt to embrace this challenge with a
focus on a highly relevant aspect of social cognition that is
very difficult to infer from an audio-visual stimulus by means of computer
algorithms: portrayed emotions.

The presentation of movie (clips) to elicit emotional responses is an
established component of emotion research, in particular with respect to more
differentiated emotional states, such as remorse or pride, that go beyond
primary emotions like fear in complexity and time-scale \cite{GL1995}. Emotion
cues in movies can be manifold: facial expressions (e.g.
smiling) \cite{Ekm1992b}, verbal cues (e.g. swearing), voice characteristics
(e.g. trembling voice) \cite{EVS+2009}, or even context cues that can be used to
reason about emotional aspects of a situation based on abstract causal
principles rather than direct perceptual cues \cite{SS2014}.  Moreover, the
emotional response to a stimulus in a real-life setting is further dependent on
additional factors. For example, observing a facial expression of sadness may
yield an emotional response of pity or satisfaction, depending on whether the
person is a friend or a punished offender of social norms. A smiling face is
not an expression of happiness when the larger context identifies it as a
strategy to avoid unpleasant social interactions (e.g. a ``fake'' smile).
Consequently, labeling portrayed emotions in feature films is a task that
requires human observers to perform complex judgements. Moreover, emotions portrayed in
movies are typically performed by actors that mimic a target expression to the
best of their abilities that does not necessarily reflect their true emotional
status.  Consequently, individual indicators of emotions may be missing,
exaggerated, or even contradictory in comparison to ``real'' emotions.

Two groups of frameworks for systematic description of emotions are
distinguished in the literature: a) schemes with discrete emotions (labels) and
b) dimensional models \cite{GW2007}.  Models using discrete emotion labels vary
considerably in the number of differentiated emotion states. Many theories
assume few basic, innate emotions \cite{Ekm1992a}, others discriminate up to
36 affective categories \cite{Sch2005}.  Dimensional models, for example the
circumplex model, locate different emotional states in a two-dimensional
space, commonly using the axes arousal and valence \cite{Rus1980}.

Current emotion assessment tools, like the Facial Action Coding System (FACS),
or the Specific Affect Coding System (SPAFF) \cite{CG2007}, attempt to combine
both approaches. However, they typically involve following complex instructions
regarding the interpretation of facial expressions or other physical and
cultural indicators of emotion \cite[p. 281]{CG2007} and, thus, require an
intensive training of observers. While these tools provide reliable procedures
for rating emotions, they are not very intuitive and consequently only
accessible to a selected group of experts. 

This study was performed by a group of observers from the same student population
that the participants in our original imaging studies were drawn
from \cite{HBI+14} (none of the observers participated in the previous study). We were aiming for a simplified procedure that combines a dimensional rating with a categorical labeling of emotions and their perceptual evidence.
The goal was to describe the emotional content of the Forrest Gump movie
stimulus as it was most likely seen by the participants, while maintaining
potential interpretation biases introduced by age, native language, or
education.

The primary purpose of these annotations is to complement our previously
released brain imaging data to enable analyses of the brain's response to
versatile emotion cues in real-life settings with real-life complexity. To this
end, dedicated annotations of the audio-movie stimulus have been performed.
Combined, these data form a two-hour high-resolution fMRI measurement for
auditory emotion cue processing from 20 participants. In order to facilitate
future studies the annotations go beyond the auditory domain and include
visual indicators of emotion. They will complement a future data release with
fMRI data recorded while participants watch the audio-visual movie.  Moreover,
these new annotations of portrayed emotions are another step towards a
comprehensive description of our reproducible movie stimulus\cite{HBI+14} and
improve its utility for independent studies with a focus on the perception of
emotion in naturalistic settings.


%But how does this physiological data
%compare to subjective perception? To produce such comparable data we annotated
%the movie Forrest Gump.

%
%In any case, reading emotional expression in another person is something we are
%confronted with every day and somehow instinctively we are all experts in it
%\cite{CG2007}. This is why we believe that a valid emotion
%assessment tool should be intuitive and usable by everyone, regardless of their
%educational background. 
%
%Hence, we aimed our study at developing an easy-to-use tool to assess emotions
%which eliminates the necessity of intensive training. Nevertheless, we
%considered it important to gain sufficiently detailed information and generate
%quality data. Therefore, we decided to combine dimensional and categorical
%measurements to describe emotions on various levels instead of focussing on its
%single aspects.
%
%Being more intuitive and freely accessible, this tool might enable and
%encourage more people interested in emotion research to enter the field,
%conduct new studies and replicate long-known findings within the more
%naturalistic setting of movies.
%
%



%Using feature films as stimuli to study the
%complex inter-play of cognitive processes involved in emotions, requires .
%
%The reliable induction of emotional states is important to understand emotional
%processes in experimental settings. One method, eliciting these emotional
%states, is the presentation of movies, especially since the interest in
%studying more differentiated emotional states is increasing. With this
%procedure researchers are able to use both, a dimensional viewpoint and
%discrete emotion perspective \cite{GL1995}.
%
%Due to the problem of eliciting emotions in the laboratory \cite{GL1995}, it would be desirable to be able to rate the portrayed emotions in
%movies and therefore improve the selection of specific audio-visual material.
%The aim of our study was to develop such a procedure to annotate the movie
%“Forrest Gump”.
%
%
%
%
%The selection of the applied emotion theory is critical for developing a
%reliable emotion rating system.
%
%Especially, a clear emotion definition and discrimination of other affective
%phenomena bears importance. In the literature attention is repeatedly drawn to
%the differentiation of mood and emotion. Moods are considered longer in
%duration, lower in intensity, with no apparent cause compared to emotion
%\cite{Sch2005,GW2007}. Since actors react primarily to proximate
%factors, e. g. the surrounding environment, other characters and their
%behavior, emotions hold the main interest in the affect assessment of movies.
%
%
%
%
%The system introduced in this study takes both approaches in consideration,
%enabling the rater to make statements about arousal and valence in binary form
%as well as adding discrete emotion labels.
%
%Another important element in consideration is through what cues emotions are
%portrayed. For instance, facial expression might be used to identify emotions
%in other humans. Ekman (1992) found evidence for up to five different
%expressions for each emotion, however the exact number of universal facial
%expressions remains unknown \cite{Ekm1992b}.
%
%Further, vocally expressed emotions are important. Ethofer and colleagues
%presented pseudowords spoken in five prosodic categories during event-related
%fMRI to investigate the discrimination of these categories based on spatial
%response patterns within the auditory cortex \cite{EVS+2009}.  Emotional
%information could be classified with specific patterns, which generalized
%across speakers.
%
%Equally important, emotions can also be inferred from situational information.
%We impute emotions in the absence of overt expressions by reasoning about the
%situation and relying on abstract causal principles rather than direct
%perceptual cues, even for expressions we have never observed nor experienced
%\cite{SS2014}. This underestimated aspect might be crucial, since not
%only the main character “Forrest Gump” sometimes lacks in emotional
%expressions. 
%
%Because of the complexity of different cues for recognising emotions, our study
%may provide evidence whether it is possible for a number of raters to perceive
%the same portrayed emotions and to what extent affects are transmitted by each
%of the different cues.
%
%One aim of our study was to develop a method which enables different
%participants to make reliable and nearly identical annotations of emotions
%shown by the actors in the movie “Forrest Gump”. 
%
%Our study might be seen as a progression of the similar methods used by Skerry
%and Saxe\cite{SS2014}. The researchers made subjects watch brief video clips
%designed to elicit the attribution of an emotional state (either positive or
%negative) in an expression or situation condition.
%
%We watched the movie in a randomised order and rated portrayed emotions in
%regard to their temporal occurrence, valence, direction, arousal as well as
%onset and offset cues. Additionally, participants were able to assign discrete
%emotion labels. Assessing jokes was also one of our interests.The raters
%assigned to one of two groups. They either watched the audio-visual version of
%the movie or listened to the audio track with audio description. We settled for
%this variation to investigate whether the absence of facial expressions and
%body language makes a crucial difference in the perception of portrayed
%emotions.
%
%Discussion
%
%In conclusion, our results indicate that is it possible to find general
%emotions in movies or other digital video data with our developed method. By
%the current state of scientific knowledge, limitations arise in rating concrete
%emotions due to the lack of ageneral system for emotion categorization. It was
%necessary to compromise between concrete emotion rating and sufficient
%consistency over all raters. Our method was developed to produce objective and
%representative data but to achieve more validated and congruent results, this
%procedure has to be repeated with more raters. The distribution in age and
%ethnic backgrounds has to be wider to produce more generalizable data. Older
%people might annotate in another way or might perceive different emotions than
%teenagers.
%
%Different cultures, socioeconomic states or different education could also lead
%to varying results.
%
%Although we covered a great range of possible hazardous variables, our rating
%system produced contradictory results especially in the rating of direction,
%valence and intensity. Because there is more than one association with the
%concepts of these words, it was possible to observe a positive and a negative
%emotion portrayed by one character at the same time between the different
%raters. It might be essential for raters to receive a special training before
%the actual rating, to gain a general understanding of used concepts. On all
%accounts,concrete definitions are essential for such sensible terms.
%
%The emotion rating was not our exclusive purpose. The rating of the jokes as
%our second interest of research was completely unaffected by the vague concepts
%of direction, intensity and valence. We intended to solely annotate the
%temporal occurrence of the jokes, because the movie “Forrest Gump” features
%many implicit jokes. However, some issues arose in their rating process leading
%to a great variation in the number of recognized jokes. Because our first
%attempt was to annotate the emotions, the primary attention was not applied to
%the jokes. In some circumstances greater attention could be achieved by the
%conduction of more rating runs or forming more groups. 
%
%Even for this annotator group the randomization of the scenes is potential to
%prevent possible sequence effects and to increase internal validity. Though,
%there has to be observance of the necessity that in this procedure we had to
%watch previous scenes due to the lacking scene splits. With this limitation it
%was not feasible to avoid some sequence effects and this results in the lack of
%internal validity.  We tried to achieve congruent results in the character
%nomination with a list of all characters occurring in the movie. But the
%results indicated that it would have been necessary to create a concrete sample
%of characters to annotate, especially involving important and insignificant
%supporting characters. This resulted in the great variation in the amount of
%annotated emotions between raters.  In summary improvements should be
%undertaken for future emotion annotations, but we made a scientific and
%profound start for this kind of method. 
%
%Conclusion
%
%Emotions portrayed by actors in a movie are scarcely examined due to the
%predominant focus on emotions elicited in subjects watching a movie.
%Nevertheless this could give an important basis to further research of the
%recognition of emotions (as listed above, emotions can be interpreted through
%facial expression, vocal expressions or situational cues) and what stimuli
%actually elicit emotions in the audience.
%
%The method is not limited to one movie, but could be used in other studies as
%well.
%
%Furthermore our data could be used to investigate whether subjects watching the
%movie actually responded with the emotion the movie was intended to produce.
%
%
%%%%%%%%%%%%%%%%%%%%%%%%%%%%%%%%%%%%%%%%%%%%%%%%%%%%%%%%%%%%%%%%%%%%%%%%%%%%%%
%%%%%%%%%%%%%%%%%%%%%%%%%%%%%%%%%%%%%%%%%%%%%%%%%%%%%%%%%%%%%%%%%%%%%%%%%%%%%%
%%%%%%%%%%%%%%%%%%%%%%%%%%%%%%%%%%%%%%%%%%%%%%%%%%%%%%%%%%%%%%%%%%%%%%%%%%%%%%
%%%%%%%%%%%%%%%%%%%%%%%%%%%%%%%%%%%%%%%%%%%%%%%%%%%%%%%%%%%%%%%%%%%%%%%%%%%%%%
%
%Inspired by the approach of Uri Hasson to investigate emotions in a more
%naturalistic way using movies we aspired to enter the field of natural
%stimulation in emotion research \cite{HNL+2004}.
%
%The study at hand was undertaken in a seminar by third semester students of
%psychology and was part of the larger project “Studyforrest” conducted by the
%Psychoinformatics laboratory at Otto-von-Guericke-University Magdeburg,
%Germany. Studyforrest collects versatile data based on the movie “Forrest
%Gump”, including measurements in fMRI, physiology, eye movement, EEG and
%others.Thereby it aims to “enable inter-disciplinary research to study the
%brain's natural behavior” (Hanke, n.d.) and to provide freely accessible data
%for researchers all over the world. 
%
%The movie “Forrest Gump” has been perceived to offer not only a wide spectrum
%of emotions, but also to depict these emotions in an easily comprehensible
%manner. 
%
%The study at hand distinguishes between displayed and perceived emotions and we
%decided to explore the former. By evaluating only the emotions presented by the
%actors (displayed emotion) we want to lay groundwork for further research
%focussing on the ones experienced while watching the movie (perceived emotion).
%
%In the course of that process, we realized that emotion rating is a very
%complex endeavour. Not only is there no universally accepted definition of what
%emotions are, but also the various dynamic and interactive reactionary systems
%emotions are expressed in make it basically impossible to reliably measure them
%\cite[p. 363]{NK2007}. According to Gray and Watson (2007),
%emotion can be measured either categorically or dimensionally \cite{GW2007}. And while both
%approaches have their scientific relevance, there are opposing opinions as to
%whether they are compatible or not \cite[p. 172f]{GW2007}.
%
%Recent emotion assessment tools like the Facial Action Coding System, FACS, or
%the Specific Affect Coding System, SPAFF \cite{CG2007}, attempt to combine both measurements.
%But they follow complex instructions regarding facial muscular activity or
%other physical and cultural indicators of emotion \cite[p. 281]{CG2007}.
%and, thus, require an intensive training. Thereby, on the one hand they provide
%a highly scientific and very reliable manner of rating emotions, on the other
%hand they consequently appear less intuitive and are only accessible to a
%selected group of experts. 
%
%In any case, reading emotional expression in another person is something we are
%confronted with every day and somehow instinctively we are all experts in it
%\cite{CG2007}. This is why we believe that a valid emotion
%assessment tool should be intuitive and usable by everyone, regardless of their
%educational background. 
%
%Hence, we aimed our study at developing an easy-to-use tool to assess emotions
%which eliminates the necessity of intensive training. Nevertheless, we
%considered it important to gain sufficiently detailed information and generate
%quality data. Therefore, we decided to combine dimensional and categorical
%measurements to describe emotions on various levels instead of focussing on its
%single aspects.
%
%Being more intuitive and freely accessible, this tool might enable and
%encourage more people interested in emotion research to enter the field,
%conduct new studies and replicate long-known findings within the more
%naturalistic setting of movies.
%
%
%%%%%%%%%%%%%%%%%%%%%%%%%%%%%%%%%%%%%%%%%%%%%%%%%%%%%%%%%%%%%%%%%%%%%%%%%%%%%%
%%%%%%%%%%%%%%%%%%%%%%%%%%%%%%%%%%%%%%%%%%%%%%%%%%%%%%%%%%%%%%%%%%%%%%%%%%%%%%
%%%%%%%%%%%%%%%%%%%%%%%%%%%%%%%%%%%%%%%%%%%%%%%%%%%%%%%%%%%%%%%%%%%%%%%%%%%%%%
%%%%%%%%%%%%%%%%%%%%%%%%%%%%%%%%%%%%%%%%%%%%%%%%%%%%%%%%%%%%%%%%%%%%%%%%%%%%%%
%
%
%Not too long ago a movement formed within the reigns of neuropsychology which
%took experimental conditions “to the streets” with ‘naturalistic stimulation’.
%A pioneer of this new direction is the ‘Study Forrest’ project which used the
%well known movie “Forrest Gump” as a stimulus to evaluate on brain activation
%during natural display of emotions. But how does this physiological data
%compare to subjective perception? To produce such comparable data we annotated
%the movie Forrest Gump.

