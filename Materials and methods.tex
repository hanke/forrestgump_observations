\section*{Materials and methods} 
%Please provide a detailed account of the methods and materials used for the study in order to allow replication by other researchers.
%Please include \begin{itemize}
%\item 
%the source of all samples, reagents, antibodies etc.
%\item 
% how samples were selectedwhat exclusions were made, if any;
%\item 
%what was being measured
%\item 
%for processed data, any software used to process the data and, where possible, the source code should be made openly available.\item 

%allowances made, if any, for controlling bias or unwanted sources of variability.
%Limitations of the datasets.\item 

%acronyms and abbreviations must be explained
%\end{itemize}

%Detailed description of the observation setup. QA statistics do not go here,
%but the "Dataset validation" section.

\subsection*{Stimulus}

The annotated stimulus was a slightly shortened version of the movie
``Forrest~Gump'' (R.~Zemeckis, Paramount Pictures, 1994). Two different
variants of this movie were annotated separately. The first one was a German
audio-description as broadcast as an additional audio track for visually
impaired listeners on Swiss public television (Koop, Michalski, Beckmann,
Meinhardt \& Benecke produced by Bayrischer Rundfunk, 2009). This audio-only
stimulus was the same as the one used in \cite[contains instructions on how to
reproduce the stimulus from the DVD release]{HBI+14}, and is largely
identical to the dubbed German soundtrack of the movie except for interspersed
narrations by a male speaker who describes the visual content of a scene. These
descriptions take place when there is no dialog, off-screen speech, or other
relevant audio-content in the movie. The second variant is the audio-visual
movie with the original dubbed German soundtrack (without narrations) and the
exact same timing (and cut scenes) as the audio-only version.

\hl{TODO: include script for blueray conversion}

\subsection*{Observers}

Annotations were created by the authors themselves as part of a practical
course on scientific observation. A total of \AVTotalRaters +\AOTotalRaters
\hl{TODO} students at the Otto-von-Guericke-University in Magdeburg, Germany
participated in this effort and received course credit.  Based on personal
preference, students assigned themselves to one of two groups tasked with the
annotation of portrayed emotion in either the audio-visual movie, or the
audio-only version. The audio-visual group consisted of \AVTotalRaters\
students (all female), the audio-only group comprised \AOTotalRaters\ students
(all male).

\hl{TODO: add a note on gender distribution in the discussion}

\subsection*{Procedure}

Annotations were performed with the help of Advene\cite{AP2005}, a dedicated
open-source video annotation tool that offers convenient navigation of
structural units of the stimulus and provided uniform position information in
milliseconds.  Each observer annotated all of 205 previously identified movie
scenes in an individual random order to minimize ``carry-over'' effects and
help observers to focus on current indicators of portrayed emotions in contrast
to more slowly changing characteristics, such as mood, or other biases
introduced by the movie plot. Scenes were used as units for annotation, because
transitions between them typically involve a change of location or fast
progression of time that make persistence of the nature of portrayed emotions
across scenes less likely.

Observers were instructed to work alone in a setting free of distractions, and
to use headphones for optimal saliency of the stimulus. There was no strict
guideline regarding the length of an annotation session, but observers were
informed to stop working for any number of breaks whenever they could no longer
guarantee a high level of attention for the task. Sessions were typically
spread over multiple days over the course of three weeks. On average, observers
required around 30 hours each to complete the annotation.

\hl{TODO we need 2-3 sentences explaining the workflow between advene and the
spreadsheet application. Most readers probably do not know advene. What did
advene contribute? How often did you look at a scene? once? twice? once per
variable?}

\subsection*{Annotation content}

Each observer collected annotations in a spreadsheet application with columns
for start time and end time of an emotion, a label of the movie character
portraying the emotion, variables for valence, arousal and direction of the
emotion, as well as an optional emotion category label, and a list of
identified indicators for the start and end of an emotion. Each of these
variables are described in more detail in the remainder of this section. In
addition, observers also recorded the index of the scene containing a
respective emotion. This variable was only used for error detection (mismatch
of emotion duration with the start and end time of a scene).

\paragraph{Start and end time}

The start time identifies the onset of the first observed indicator of an
emotion. Likewise, the end time corresponds to the offset of an emotion,
defined as the time were no evidence indicating an emotion is present anymore.
Observers were instructed to aggregate short segments of uniform emotional
episodes into a single annotation. For example, if a character was happy and
smiling throughout a long period in a scene, the annotation would span all the
time from the beginning to the end of the smile even if the character's face was
not visible the entire period, as long as no evidence of a change of emotion
was found.  All times are reported in seconds. Although Advene offers time
stamps with the precision of the movie frame rate (25 frames per second),
pre-tests revealed that the temporal accuracy of human observers for complex
emotion indicators, like a developing facial expression, is much coarser. In
order to reduce the chance of unrecoverable typos in the time stamps, observers
were instructed to record timing information as the integer portion of the time
in seconds only.

\paragraph{Onset and offset indicators}

Indicators of emotion onset and offset further characterize the nature of a
portrayed emotion. Onset indicators describe what kind of evidence for an
emotion was detected in the movie stimulus. We distinguished facial
expressions, body language cues, context information, verbal and non-verbal
audio cues. Observers could record multiple onset indicators. Despite the term
``onset'' these indicators did not all have to be present at the very beginning
of an emotional episode. For example, an extended period of sadness could start
with a facial expression and later on add a congruent body language cue. In
that case the respective labels were aggregated in the "onset indicators"
column. In the audio-movie stimulus the narrator was frequently the only source
of information regarding the emotional state of a character, hence we included
a dedicated category for this scenario.

In addition to onset indicators, observers also recorded the type of evidence
for the end of an emotional episode. Four conditions were distinguished: 1) the
change from one emotion to another, 2) entering a neutral emotional state, 3) a
character leaving an ongoing scene with no further evidence for its emotional
state, and 4) the end of a scene.  Table \ref{tab:onoffset_indicators}
describes all indicator categories.

\begin{table}
  \centering
  \begin{tabular}{lp{10cm}}
    \textbf{Label} & \textbf{Description} \\
    \\\hline\\
    \textit{Onset}\\\\
    AUDIO & Non-verbal audio cues, feelings expressed without words, for example laughing, crying, wheezing, sighing\\
    CONTEXT & Emotion portrayed through context information and not directly by a character, for example a social rejection of an individual by a group without visible emotional response\\
    GESTURE & Body language cues \\
    FACE & Facial expressions \\
    NARRATOR & Description of an emotion given by the narrator (only for audio stimulus)\\
    VERBAL & Emotion expressed through words either as a direct description (e.g. a self-report) or by means of a particular choice of words (e.g. harsh language)\\
    \\
    \textit{Offset}\\\\
    EMOCHANGE & Present emotion is replaced by another, different emotion \\
    EMOFADED & Emotion fades with a change to an emotion neutral state \\
    CHARLEFT & Character leaves a scene (no further evidence for an emotion)\\
    SCENEEND & Scene transition with a change of time or location indicates the end of an emotion\\

  \end{tabular}
  \caption{Categories for cues indicating the onset or offset of an emotion.}
  \label{tab:onoffset_indicators}
\end{table}

\hl{TODO: are there annotations that do not contain indicators specifications?}


\paragraph{Character label}

This label identifies which movie character is portraying an emotion. Observers
were provided with a list of character labels that was derived from an
annotation of the movie's dialog in order to achieve a consistent labeling.
However, some characters portray emotions but have no lines of dialog, in other
instances observer chose inconsistent labels. During data curation character
labels were therefore manually consolidated into a set of 35 categories that
uniformize labeling across observers. The full list of character labels is shown in table
\ref{tab:characters}. The 35 character labels were derived using the following
rules: a) main characters and famous/familiar characters receive an individual
labels, as do the two narrator voices; b) relatives of main characters are
labeled according to their relation; c) all other characters are consolidated
into generic categories that preserve gender, age (children, adults, or seniors),
and number (individual, or group).

\begin{table}
  \centering
  \begin{tabular}{p{4cm}p{4cm}p{3.6cm}p{4cm}}
    \textbf{Character label} & \textbf{Description} & \textbf{Character label} & \textbf{Description} \\
\\\hline\\

\multicolumn{4}{l}{\textit{Main characters}} \\\\
BUBBA     & Forrest's army pal & FORRESTJR & young Forrest Gump \\
DAN       & Lt.~Dan & JENNY     & Forrest's love \\
FORREST   & adult Forrest Gump &           & \\
\\\hline\\

\multicolumn{4}{l}{\textit{Special}} \\\\
FORRESTVO & voice-over narrator Forrest & NARRATOR & audio-description narrator \\
\\\hline\\

\multicolumn{4}{l}{\textit{Famous characters}} \\\\
 ABBIEHOFFMAN     & activist (lookalike actor)                   & NEILARMSTRONG    & first man on the moon (orig.~footage) \\ 
 BOBHOPE          & entertainer (orig.~footage)                  & PRESIDENTJOHNSON & US-president (orig.~footage) \\
 DICKCAVETT       & talk master (orig.~footage)                  & PRESIDENTKENNEDY & US-president (orig.~footage) \\
 ELVISPRESLEY     & The King (orig.~footage and lookalike actor) & PRESIDENTNIXON   & US-president (orig.~footage)  \\
 JOHNLENNON       & Beatle (orig.~footage)                       & ROBERTKENNEDY    & Brother of J.~F.~Kennedey (orig.~footage) \\                  & \\
\\\hline\\

\multicolumn{4}{l}{\textit{Relatives of main characters}} \\\\
BUBBASGRGRANDMA   & Bubba's great-grandmother & JENNYSGRANDMA & Jenny's grandmother\\
BUBBASGRGRGRANDMA & Bubba's great-great-grand\-mother & MRSBLUE       & Bubbas's mother\\
BUBBASYOUNGBROTHER      & Bubba's young brother & MRSGUMP       & Forrest's mother \\
JENNYSDAD               & Jenny's father &              & \\
\\\hline\\

\multicolumn{4}{l}{\textit{Generic character categories}} \\\\
CHILDREN  & group of children & MEN      & group of men \\
GIRL      & individual girl & OLDMEN   & group of old men \\
BOY       & individual boy & WOMAN    & individual woman \\
OLDERBOYS & group of older boys & OLDWOMAN & old woman\\
MAN       & individual man & WOMEN    & group of women \\
OLDMAN    & individual old man & CROWD    & group of people with mixed gender \\
 \\\hline
  \end{tabular}
  \caption{Consolidated character labels. Any annotation of a portrait emotion
  is associated with exactly one of these 36 labels. While main characters,
  their relatives, and famous characters have individual labels, all remaining
  characters in the movie are aggregated into generic categories that preserve
  information on gender, ange group and number.}
  \label{tab:characters}
\end{table}

\paragraph{Valence, arousal and direction}

Binary variables were used to record the valence of an emotion (positive vs.
negative), the state of arousal of the character portraying it (high vs. low),
and whether the emotion is directed towards the character itself or another one
(feeling pity for somebody). Together, these three indicator variable offer a
coarse categorization of observed emotion in the spirit of dimensional models
of emotion.

Arousal was considered a global indicator of the emotional status of a
character, i.e. for any given time in the movie annotations either indicate a
high level of arousal for a character or a low level, but never both at the
same time. Unlike arousal, valence and direction indicators were considered
non-exclusive, i.e. a character could show signs of positive and negative
emotions at the same time. Likewise a character could exhibit self-directed
emotions, and emotions directed towards others simultaneously.

\paragraph{Emotion labels}

Observers were instructed to assign one of 22 discrete emotion labels. Emotion
categories were derived from \cite{Ort1990} and concrete definitions were
developed based on the Specific Affect Coding System \cite{CG2007}. The
complete list of emotions is shown in table \ref{tab:emotion_categories}.
Pre-tests using a larger and more detailed set of emotion categories reveal
weak agreement between observers for a forced categorization of emotional
episodes. Based on these results we decided to make this part of the annotation
optional and observers were instructed to assign an emotion label only if they
saw a ``perfect'' match with any of the categories.

\hl{TODO: give list of max inter-rater reliability for labels}

\begin{table}
  \centering
  \begin{tabular}{lp{10cm}}
    \textbf{Label} & \textbf{Description} \\
    \\\hline\\
    ADMIRATION & Appreciation of another person or object \\
    ANGERRAGE & Response to injured respect or autonomy of a person by another one \\
    COMPASSION & Concern in someone's suffering \\
    CONTEMPT & Disdain of another person or object \\
    DISAPPOINTMENT & When expectations were not met \\
    FEAR & Triggered by a dangerous situation \\
    FEARCONFIRMED & \hl{TODO: missing description}\\
    GLOATING & To be happy because another person suffers \\
    GRATIFICATION & Triggered by a positive event which one owes by themselves  \\
    GRATITUDE & Triggered by a positive event which one owes by another person \\
    HAPPINESS & Pleasant and joyful condition \\
    HAPPYFOR & concern in happiness of someone else \\
    HATE & Rejection of another person, intense dislike, or animosity \\
    HOPE & Positive expectation for a person or special event in the future \\
    LOVE & Affection to another person \\
    PRIDE & Positive reflection of accomplishments and status \\
    RELIEF & A worry/negative expectation do not come to pass \\
    REMORSE & Triggered by a negative event initiated by themselves \\
    RESENTMENT & Response to perceived unfair treatment \\
    SADNESS & Emotional pain after loss or misfortune \\
    SATISFACTION & When expectations get fulfilled \\
    SHAME & Dissatisfaction with one's own behavior, because norms are violated \\
  \end{tabular}
  \caption{List of 22 emotion categories observer could (optionally) use to
    futher qualify the nature of a portrayed emotion. \hl{TODO: doublecheck descriptions}}
  \label{tab:emotion_categories}
\end{table}



\subsection*{Data file organization}

In order to maximize utility of the annotations they are released in full, and
not only as aggregate information. The annotations from each observer are
available in an individual plain text file with comma-separated value (CSV)
markup. This file format facilitates import and further processing in virtually
any software application that supports data in tabular form. The filenames
indicate the annotated stimulus type (prefix \texttt{av} for the audio-visual
movie and \texttt{ao} for the audio-only movie), as well as the observer
identity.

Each file contains nine columns that correspond to the annotation properties
described above. \texttt{start})~start and \texttt{end})~end time of an emotion
episode, reported in seconds from the start of the movie; \texttt{character})~a
single character label from the set shown in table \ref{tab:characters}
indication the character portraying the emotion; \texttt{arousal})~state of
arousal indicated with the labels \texttt{LOW} or \texttt{HIGH};
\texttt{valence})~emotional valence indicated with the label \texttt{POS} or
\texttt{NEG}; \texttt{direction})~direction of the emotion indicated with the
labels \texttt{SELF} or \texttt{OTHER}; \texttt{emotion})~a space-separate list
of emotion labels from the set shown in table \ref{tab:emotion_categories};
\texttt{oncue})~a space-separated list of onset indicators for a portrayed
emotion from the set shown in table \ref{tab:onoffset_indicators}; and
\texttt{offcue})~a space-separated list of offset indicators from the set also
shown in table \ref{tab:onoffset_indicators}.  The first six columns contain
mandatory information that is present for all annotations. The remaining
columns contain optional attributes and can be empty.

\subsection*{Time stamp conversion}

To be able to use the provided emotion annotations for an analysis of
functional data (fMRI, respiratory and cardiac trace) acquired using the
procedure described in~\cite{HBI+14} timing information has to be converted,
because data were recorded in eight sessions using partially overlapping
segments of the movie. Table \ref{tab:timing} specifies the location of these
segments with respect to the timing of the unsegmented movie used for emotion
annotation. Subtracting movie segment start times from annotation time stamps
yields valid relative time stamps with respect to the time series of the
available functional data.

\begin{table}
  \centering
  \begin{tabular}{lrrrr}
    \textbf{\#}  & \textbf{Start}  & \textbf{End}    & \textbf{Duration}& \textbf{Boundary} \\
  \\\hline\\
   0  & 0.0    & 902.0  & 902.0  & 891.2    \\
   1  & 886.0  & 1768.0 & 882.0  & 1759.2   \\
   2  & 1752.0 & 2628.0 & 876.0  & 2618.8   \\
   3  & 2612.0 & 3588.0 & 976.0  & 3578.5   \\
   4  & 3572.0 & 4496.0 & 924.0  & 4488.0   \\
   5  & 4480.0 & 5358.0 & 878.0  & 5349.2   \\
   6  & 5342.0 & 6426.0 & 1084.0 & 6418.2   \\
   7  & 6410.0 & 7086.0 & 676.0  & 7085.5   \\
  \end{tabular}

  \caption{Position of the fMRI stimulus segments with respect to the movie
    annotation time. The last column lists the position of the scene boundary
    used as a reference for the segment transition. All times are in seconds
    and refer to the full (unsegmented) stimulus, which is a shorted version of
    the original movie \cite{HBI+14}.}

  \label{tab:timing}
\end{table}

%\paragraph{Jokes}
%Finally, we also annotated jokes during the movie, defining them as scenes or lines that were intended to be funny and make the audience laugh. Jokes were annotated only in their duration and marked with the abbreviation JOK in the emotion column. 
%

% Raters
% 
% The sample consisted of 14 psychology students in the third semester of the
% Bachelor from the Otto-von-Guericke university in Magdeburg . The participants
% assigned to a group which either annotated the audio-visual version of the
% movie or listened only to the audio track with additional audio description for
% visual impaired people. … students worked in the first group, consisting of …
% female and … male participants ranging from .. to.. years. … male raters
% participated in the audio-based group. They were … to … years old. The study
% was conducted during a graded course dealing with experimental observation.
% Students were rewarded with 2 CP for completing the seminar. 
% 
% Setup
% 
% First each rater was obliged to download and install the programme advene\cite{AP2005}, a
% software to share annotations about digital video documents from the following
% website: http://liris.cnrs.fr/advene/download.html.
% 
% The unitary version of the movie Forrest Gump was imported to this programme.
% We decided to set the time format uniformly to seconds.milliseconds. The
% program was chosen to ensure consistent basic conditions while watching the
% movie.
% 
% For the annotation process we used a numbered list of all scenes with start and
% end point, occurring characters and setting. Every rater randomized the scenes
% using a self chosen random number generator to obtain objectivity and to avoid
% sequence effects.
% 
% Rating conditions
% 
% Every rater was instructed to annotate the movie in a relaxed state without
% time pressure. Furthermore, the session needed to be neutral and with a low
% rate of stimuli, to avoid any distractions, neither by technical devices nor
% other programmes. Hence every rater was supposed to use headphones. Any
% communication between the participants should be prevented by working
% separately during the annotation process to make sure the results were
% independent.
% 
% Each rater annotated the whole entirety of the scenes. The scene number refers
% to the standardized number of the presented movie version shown in advene\cite{AP2005}.
% Every participant created an individual table with their preferred program to
% work with.
% 
% Rating Content 
% 
% Every annotation consisted of the following elements: the scene number;
% temporal occurrence; the name of the character; dimensions of valence, arousal
% and direction of the emotion, onset and offset cues and jokes.
% 
% To investigate time course of jokes we rounded beginning and ending of each
% joke to whole seconds. All of the other columns were left blank. 
% 
% Scene number and time
% 
% The scene number was used for improved detection of mistakes in the annotation
% of the time occurrence. Beginning and ending of the displayed emotions was
% rounded to whole seconds.
% 
% Valence, arousal and direction
% 
% The rating of the three dimensions was feasible by using a binary code or
% shortcut. Positive valence might be abbreviated with “pos”, “p” or 1, whereas
% “neg”,”n” or 0 stood for negative valence. High arousal was synonymous to
% “high”,”h” or 1, low arousal to “low”,”l” or 0. Finally, “self”,”s” or 0 was a
% shortcut for self-directed emotions and “other”,”o” or 1 was used for emotions,
% which were elicited or directed towards other people. 
% 
% Emotion labels
% 
% Each participant was also able to assign discrete emotion labels using the
% Cognitive Structure of Emotions \cite{Ort1990} or leave specific comments.
% Based on the Specific Affect Coding System \cite{CG2007} we developed
% definitions for the emotions in the named model in order to achieve a
% collective comprehension. A final list of our discrete emotion labels is
% attached at the end of our research paper. 
% 
% Onset and offset indicators
% 
% Onset and offset indicators define what kind of cues were used to decide on the
% portrayed emotion.
% 
% Likewise every annotation starts with an onset indicator and ends with an
% offset indicator. 
% 
% Onset indicators might contain mimic (m), gesture and body language (g),
% context (c), verbal (v) and non-verbal audio cues (a). Participants only rating
% the audiotrack of the movie could also use a narrator cue (n) as an indicator
% for emotions.
% 
% Context cues refer to any portrayed emotion that is inferred from the situation
% rather than perceptual cues.
% 
% Offset indicators included emotion change (e), fading of an emotion (f), the
% departure of a character (l) and scene ends (s).
% 
% %%%%%%%%%%%%%%%%%%%%%%%%%%%%%%%%%%%%%%%%%%%%%%%%%%%%%%%%%%%%%%%%%%%%%%%%%%%%
% %%%%%%%%%%%%%%%%%%%%%%%%%%%%%%%%%%%%%%%%%%%%%%%%%%%%%%%%%%%%%%%%%%%%%%%%%%%%
% %%%%%%%%%%%%%%%%%%%%%%%%%%%%%%%%%%%%%%%%%%%%%%%%%%%%%%%%%%%%%%%%%%%%%%%%%%%%
% %%%%%%%%%%%%%%%%%%%%%%%%%%%%%%%%%%%%%%%%%%%%%%%%%%%%%%%%%%%%%%%%%%%%%%%%%%%%
% 
% The group of raters consisted of XXX students (XXX male, XXX female) of
% psychology in the third semester age XXX to XXX ( mean=XXX). The process of
% annotation, preparation, raising and processing of the data will be described
% in the following.
% 
% To reach the final state of our annotation tool several test runs were
% necessary. We started by using the categorical approach in order to classify
% the most important emotions and differentiate them clearly from one another. We
% decided to adopt the “global structure of emotions” \cite{Ort1990}.
% Others, like Ekman and Friesen’s well-established model of the six basic
% emotions\cite{EF1975} seemed too reductive for our purpose. The catalogue
% incorporates 22 different emotions, which are listed below. Knowing  that each
% emotion is a subjective construct, we added individual definitions based on
% Ortony’s explications and, if necessary, our own experience. By that we hoped
% to reduce the heterogeneity of interpretation and thus to increase the
% inter-rater reliability. 
% 
% After testing our first concept twice, we realized that some explicitly
% depicted emotions were not classifiable within our emotion catalogue. We
% decided to widen the rating possibilities by adding the three dimensional
% categories valence (positive/negative), arousal (high/low) and direction
% (self/other). In the following, every depicted emotion had to be rated
% dimensionally, using binary coding, but the previously binding categorical tag
% was now optional. In that way it was no longer necessary to fit the depicted
% emotions into the offered categories if the raters did not find a matching
% description. The additional tagging of specific emotions nevertheless broadens
% the informative value of the dimensional descriptions and was therefore
% continued. In effect this method turned out positively as inter-rater
% reliability improved.
% 
% During the process of annotation the raters were urged to work alone, and
% minimize distractions by technical devices (i.e. mobile phones) and by their
% surroundings. They were therefore also required to use headphones. In addition
% communication among raters about details of the annotation was not allowed.
% 
% One set of 4/5 raters annotated an audio file of the movie, thereby
% contributing data comparible to MRI and EEG data gathered during previous
% phases of the studyforrest project from participants only listening to the
% audio version of the movie.
% 
% In order to structure the annotation process we decided to analyze the movie
% scenewise. Every rater annotated all 240 scenes in a randomized order so that
% any bias caused by the movie plot could be prevented. All raters used the same
% license-free program, Advene\cite{AP2005}, to play the video material while annotating the
% scenes, in this way, making sure everyone referred to the same timeline.
% Annotations were made in spreadsheets, each rater using the program of his
% choice but applying a consistent composition of the table (see table).
% 
% The subgroup of raters attending to the audio version followed the same
% instruction manual, the only exception being the use of a different file and
% covering/closing the visual display in Advene.
% 
% Furthermore we annotated the character expressing the emotion, abbreviating the
% names of the six most frequent characters, and the duration of the shown
% emotion. For every annotation the raters had to state the precise time an
% emotional depiction started and ended in seconds.  For additional infomation on
% the circumstances emotions were expressed in, we implemented onset- and
% offset-indicators  that started or ended an emotion (for complete list see
% indicators). The time that we first/last perceived an indicator, had to mark
% the exact beginning/end of the emotion. If there was more than one valid
% indicator present they were listed in alphabetical order and separated by a
% blank character (e.g. context and mimic onset-cue together were marked as “c
% m”). This way we were not limited to only one or two response systems, but were
% able to describe expressed emotion on various levels. The complete list and
% definitions of all the indicators used can be found in the instruction manual.
% 
% Finally, we also annotated jokes during the movie, defining them as scenes or
% lines that were intended to be funny and make the audience laugh. Jokes were
% annotated only in their duration and marked with the abbreviation JOK in the
% emotion column.
% 
% %%%%%%%%%%%%%%%%%%%%%%%%%%%%%%%%%%%%%%%%%%%%%%%%%%%%%%%%%%%%%%%%%%%%%%%%%%%%%
% %%%%%%%%%%%%%%%%%%%%%%%%%%%%%%%%%%%%%%%%%%%%%%%%%%%%%%%%%%%%%%%%%%%%%%%%%%%%%
% %%%%%%%%%%%%%%%%%%%%%%%%%%%%%%%%%%%%%%%%%%%%%%%%%%%%%%%%%%%%%%%%%%%%%%%%%%%%%
% %%%%%%%%%%%%%%%%%%%%%%%%%%%%%%%%%%%%%%%%%%%%%%%%%%%%%%%%%%%%%%%%%%%%%%%%%%%%%
% 
% As a first step we searched for a fitting model of cataloguing emotions. We
% ultimately used the 22 emotion labels according to the model of Ortony, Clore
% and Collins (citation 1) but additionally collected dichotomous data for the
% valence, arousal and direction of the elicited emotion, to avoid the problem of
% rating ambiguous emotions. To ensure the interrater reliability we developed a
% standardized annotation manual (attachment 1). 
% 
% For the purpose of annotation we used the open source annotation-program
% 'Advene'\cite{AP2005} which we loaded with the research cut of ‘Forrest Gump’ and the
% previously raised stimulus annotations from the Study Forrest project (citation
% 2). The movie was then displayed scene by scene whereas the individual segments
% where randomized by a random number generator.
% 
% The raters were separated into two groups: Group one annotated the audiovisual
% version of the movie; group two raised the information from an audio track for
% visually impaired listeners.
% 
% Both groups were advised to operate strictly after the provided
% annotation-manual.  The rating process itself was spread out over three weeks;
% the raters were advised to work individually in a distraction free environment
% of their choosing with a neutral emotional state.
% 
% The annotation itself could be written in any table or a text-processing
% program, as long as raters adhered to the standardized mask stipulated in the
% manual and didn’t exchange information relevant to the rating process. 
% 
% The separate results where then cumulated into one single dataset for
% evaluation.
